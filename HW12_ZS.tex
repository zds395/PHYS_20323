%%%
\documentclass[12pt]{article}
\usepackage{epsfig}
\usepackage{times}
\usepackage{amsmath}
\renewcommand{\topfraction}{1.0}
\renewcommand{\bottomfraction}{1.0}
\renewcommand{\textfraction}{0.0}
\setlength {\textwidth}{6.6in}
\hoffset=-1.0in
\oddsidemargin=1.00in
\marginparsep=0.0in
\marginparwidth=0.0in                                                                               
\setlength {\textheight}{9.0in}
\voffset=-1.00in
\topmargin=1.0in
\headheight=0.0in
\headsep=0.00in
\footskip=0.50in                                         
\setcounter{page}{1}
\begin{document}
\def\pos{\medskip\quad}
\def\subpos{\smallskip \qquad}
\newfont{\nice}{cmr12 scaled 1250}
\newfont{\name}{cmr12 scaled 1080}
\newfont{\swell}{cmbx12 scaled 800}
%%%%%%%%%%%%%%%%%%%%%%%%%%%%%%%%%%%%%%%%%%%%%%%%%%%%%%%%%%%%
%     DO NOT CHANGE ANYTHING ABOVE THIS LINE
%%%%%%%%%%%%%%%%%%%%%%%%%%%%%%%%%%%%%%%%%%%%%%%%%%%%%%%%%%%%
%     DO NOT CHANGE ANYTHING ABOVE THIS LINE
%%%%%%%%%%%%%%%%%%%%%%%%%%%%%%%%%%%%%%%%%%%%%%%%%%%%%%%%%%%%
%     DO NOT CHANGE ANYTHING ABOVE THIS LINE
%%%%%%%%%%%%%%%%%%%%%%%%%%%%%%%%%%%%%%%%%%%%%%%%%%%%%%%%%%%%

\begin{center}
{\large
\textbf{PHYS  20323/60323: Fall 2020 - LaTeX Example}}\\
\end{center}

\vskip0.1in
\begin{enumerate}      % first begin-----------]
\item Consider a particle confined in a two-dimensional infinite square well
\begin{equation*}
V(x,y)=
\begin{cases}
0, \,\,\,\quad \textrm{if} \, 0 \leq x \leq a, \, 0<y<a\\
\infty, \quad \textrm{otherwise}$$
\end{cases}
\end{equation*}
\\The eigenfunctions have the form:
\begin{equation*}
\\ \Psi(x,y) = \frac{2}{a} \textrm{sin}(\frac{n\pi x}{a}) \textrm{sin}(\frac{m\pi y}{a})
\end{equation*}
\\with the corresponding energies being given by:
\begin{equation*}
\\E_{nm}=(n^2+m^2)\,\frac{\pi^2 \hbar^2}{2ma^2}
\end{equation*}
\begin{enumerate}     % second begin
\item (5 points) What are the levels of degeneracy of the five lowest energy values?
\item (5 points) Consider a perturbation given by:
\begin{equation*}
\\\hat{H}'=a^2V_{0}\delta(x-\frac{a}{2})\, \delta(y-\frac{a}{2})
\end{equation*}
\\Calculate the first order correction to the ground state energy.
\end{enumerate}       % second end
\vskip0.1in
\item \textbf{The following questions refer to stars in the Table below.}     
\\Note: There may be multiple answers.

\begin{tabular}{|l|c|c|c|c|c|}\hline
Name & Mass & Luminosity & Lifetime & Temperature & Radius\\\hline
Zeta & $60\, M_{sun}$   &  $10^6 \,L_{sun}$ & $8.0\times 10^5$ years &  & \\\hline
Epsilon & $6.0\, M_{sun}$ & $10^3 \,L_{sun}$  &   & 20,000 K & \\\hline
Delta & $2.0\, M_{sun}$  &  & $5.0\times 10^8$ years &  & $2\, R_{sun}$ \\\hline
Beta & $1.3\, M_{sun}$ & $3.5\, L_{sun}$ &  &  &  \\\hline
Alpha & $1.0\, M_{sun}$ &   &   &   & $1\, R_{sun}$   \\\hline
Gamma & $0.7\, M_{sun}$ &  &  $4.5\times 10^{10}$ years & 5000 K  &  \\\hline
\end{tabular}\vskip 0.2in

\begin{enumerate}     % third begin
\item (4 points) Which of these stars will produce a planetary nebula at the end of their life.
\vskip0.3in
\item (4 points) Elements heavier than \textit{Carbon} will be produced in which stars.                                 
\end{enumerate}        % third end----]        
\end{enumerate}        % first end


\end{document}
